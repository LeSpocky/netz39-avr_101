%%%%%%%%%%%%%%%%%%%%%%%%%%%%%%%%%%%%%%%%%%%%%%%%%%%%%%%%%%%%%%%%%%%%%%%%
%%% documentclass and packages
%%%%%%%%%%%%%%%%%%%%%%%%%%%%%%%%%%%%%%%%%%%%%%%%%%%%%%%%%%%%%%%%%%%%%%%%
\RequirePackage{atbegshi}           % workaround for newer PGF versions
\documentclass{beamer}
% https://sourceforge.net/tracker/index.php?func=detail&aid=1848912&group_id=92412&atid=600660
\usepackage{lmodern}
\usepackage[T1]{fontenc}
\usepackage[utf8]{inputenc}
\usepackage{textcomp}
\usepackage[ngerman]{babel}
\usepackage[babel,english=american,german=guillemets]{csquotes}	% french
\usepackage{microtype}

% colors for listings
\definecolor{lightergray}{gray}{.95}
\definecolor{darkblue}{rgb}{0,0,0.5}
\definecolor{darkgreen}{rgb}{0,0.5,0}
\definecolor{darkred}{rgb}{0.5,0,0}
\definecolor{darkerblue}{rgb}{0,0,0.4}
\definecolor{darkergreen}{rgb}{0,0.4,0}
\definecolor{darkerred}{rgb}{0.4,0,0}

%\usepackage{listings}
%\lstloadlanguages{HTML,XML}
%\lstset{
%    basicstyle=\ttfamily\small\mdseries,
%    keywordstyle=\bfseries\color{darkblue},
%    identifierstyle=,
%    commentstyle=\color{darkgray},
%    stringstyle=\itshape\color{darkred},
%    frame=none,
%    showstringspaces=false,
%    tabsize=4,
%    backgroundcolor=\color{lightergray},
%}

%%%%%%%%%%%%%%%%%%%%%%%%%%%%%%%%%%%%%%%%%%%%%%%%%%%%%%%%%%%%%%%%%%%%%%%%
%%% preparations for beamer
%%%%%%%%%%%%%%%%%%%%%%%%%%%%%%%%%%%%%%%%%%%%%%%%%%%%%%%%%%%%%%%%%%%%%%%%
\useinnertheme{default}
\useoutertheme{infolines}
%\usecolortheme[rgb={0.28,0.37,0.52}]{structure}
\usecolortheme[rgb={0.18,0.23,0.33}]{structure}
%\usecolortheme{beaver}
\usefonttheme{structurebold}

%%% Ränder vergrößern für's Café Central
\setbeamersize{text margin left=1.2cm}
\setbeamersize{text margin right=1.2cm}

%%%%%%%%%%%%%%%%%%%%%%%%%%%%%%%%%%%%%%%%%%%%%%%%%%%%%%%%%%%%%%%%%%%%%%%%
%%% images
%%%%%%%%%%%%%%%%%%%%%%%%%%%%%%%%%%%%%%%%%%%%%%%%%%%%%%%%%%%%%%%%%%%%%%%%
%\pgfdeclareimage[height=0.75\paperheight]{strasse}{strasse}
%\pgfdeclareimage[height=0.75\paperheight]{arbeitsraum}{arbeitsraum}
%\pgfdeclareimage[height=0.75\paperheight]{lounge}{lounge}
%\pgfdeclareimage[height=0.75\paperheight]{mate}{mate}
%\pgfdeclareimage[height=0.75\paperheight]{regal}{regal}
%\pgfdeclareimage[height=0.75\paperheight]{werkstatt}{werkstatt}

%%%%%%%%%%%%%%%%%%%%%%%%%%%%%%%%%%%%%%%%%%%%%%%%%%%%%%%%%%%%%%%%%%%%%%%%
%%% title, author, date
%%%%%%%%%%%%%%%%%%%%%%%%%%%%%%%%%%%%%%%%%%%%%%%%%%%%%%%%%%%%%%%%%%%%%%%%
\title{AVR101}
\subtitle{Kapitel 1 -- Einführung}
\author{Netz39 e.\,V.}
\institute{\url{http://www.netz39.de/}}
\date{2013-05-13}
\subject{subj}
\keywords{AVR, Workshop}

%%%%%%%%%%%%%%%%%%%%%%%%%%%%%%%%%%%%%%%%%%%%%%%%%%%%%%%%%%%%%%%%%%%%%%%%
%%% document
%%%%%%%%%%%%%%%%%%%%%%%%%%%%%%%%%%%%%%%%%%%%%%%%%%%%%%%%%%%%%%%%%%%%%%%%
\begin{document}

\begin{frame}
	\titlepage
\end{frame}

\begin{frame}{Überblick}
    \tableofcontents
\end{frame}

\section{Warum?}

\subsection{Computer}

\begin{frame}{Wie funktioniert ein Rechner}
    \begin{itemize}
        \item CPU
        \item Register
        \item \dots
    \end{itemize}
\end{frame}

\begin{frame}{Assembler und C}
    \begin{itemize}
        \item hardwarenah
        \item \dots
    \end{itemize}
\end{frame}

\subsection{Mikrocontroller}

\begin{frame}{Wo überall?}
    \begin{itemize}
        \item Industrie
        \item Haushalt
        \item \dots
        \pause
        \item Auto
    \end{itemize}
\end{frame}

\begin{frame}{Was kann ich damit basteln?}
    \begin{itemize}
        \item Blinkezeugs
        \item Quadrocopter
        \item \dots
    \end{itemize}
\end{frame}

\section{Mit welcher Hardware?}

\subsection{AVR Mikrocontroller}

\begin{frame}{ATtiny und ATmega}
    add pictures here \dots
\end{frame}

\subsection{Schaltungen}

\begin{frame}{Steckbrett}
    Bild?
\end{frame}

\begin{frame}{Starterkit}
    Bild und Link
\end{frame}

\begin{frame}{Developerkitz}
    Mit Preisen!!
\end{frame}

\begin{frame}{Arduino}
    Ja, das geht!
\end{frame}

\section{Mit welcher Software?}

\begin{frame}{AVR Studio}
    This is M\$ Windoof, who uses this shit?
\end{frame}

\begin{frame}{Linux}
    \emph{CLI rulez!}
\end{frame}

\section{Kontakt}

\begin{frame}{Kontakt}
    \begin{center}
        \begin{description}[Twitter/identi.ca]
            \item[WWW] \url{http://www.netz39.de/}
            \item[Twitter/identi.ca] @netz39
            \item[E-Mail] kontakt@netz39.de
            \item[Mailingliste] list@netz39.de
            \item[IRC] \#netz39 auf freenode
        \end{description}
    \end{center}
\end{frame}

\appendix

\section{Lizenz}

\begin{frame}{Lizenz}
    % \begin{block}{Bilder}
        % Die Bilder sind unter folgender Creative Commons-Lizenz
        % veröffentlicht: \emph{Namensnennung-Keine kommerzielle
        % Nutzung-Weitergabe unter gleichen Bedingungen 3.0}, (CC-BY-NC-SA
        % 3.0).
    % \end{block}
    \begin{block}{Folien}
        Die Folien sind freigegeben unter \emph{Creative Commons
        Namensnennung-Weitergabe unter gleichen Bedingungen 3.0
        Deutschland Lizenz}, (CC-BY-SA 3.0). Download unter:
        \url{https://github.com/netz39/avr_101}
    \end{block}
\end{frame}

\end{document}
