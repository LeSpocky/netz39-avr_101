%%%%%%%%%%%%%%%%%%%%%%%%%%%%%%%%%%%%%%%%%%%%%%%%%%%%%%%%%%%%%%%%%%%%%%%%
%%% documentclass and packages
%%%%%%%%%%%%%%%%%%%%%%%%%%%%%%%%%%%%%%%%%%%%%%%%%%%%%%%%%%%%%%%%%%%%%%%%
\RequirePackage{atbegshi}           % workaround for newer PGF versions
\documentclass{beamer}
% https://sourceforge.net/tracker/index.php?func=detail&aid=1848912&group_id=92412&atid=600660
\usepackage{lmodern}
\usepackage[T1]{fontenc}
\usepackage[utf8]{inputenc}
\usepackage{textcomp}
\usepackage[ngerman]{babel}
\usepackage[babel,english=american,german=guillemets]{csquotes}	% french
\usepackage{microtype}

% colors for listings
\definecolor{lightergray}{gray}{.95}
\definecolor{darkblue}{rgb}{0,0,0.5}
\definecolor{darkgreen}{rgb}{0,0.5,0}
\definecolor{darkred}{rgb}{0.5,0,0}
\definecolor{darkerblue}{rgb}{0,0,0.4}
\definecolor{darkergreen}{rgb}{0,0.4,0}
\definecolor{darkerred}{rgb}{0.4,0,0}

\usepackage{listings}
\lstloadlanguages{C}
\lstset{
    basicstyle=\ttfamily\small\mdseries,
    keywordstyle=\bfseries\color{darkblue},
    identifierstyle=,
    commentstyle=\color{darkgray},
    stringstyle=\itshape\color{darkred},
    frame=none,
    showstringspaces=false,
    tabsize=4,
    backgroundcolor=\color{lightergray},
}

%%%%%%%%%%%%%%%%%%%%%%%%%%%%%%%%%%%%%%%%%%%%%%%%%%%%%%%%%%%%%%%%%%%%%%%%
%%% preparations for beamer
%%%%%%%%%%%%%%%%%%%%%%%%%%%%%%%%%%%%%%%%%%%%%%%%%%%%%%%%%%%%%%%%%%%%%%%%
\useinnertheme{default}
\useoutertheme{infolines}
%\usecolortheme[rgb={0.28,0.37,0.52}]{structure}
\usecolortheme[rgb={0.18,0.23,0.33}]{structure}
%\usecolortheme{beaver}
\usefonttheme{structurebold}

%%% Ränder vergrößern für's Café Central
\setbeamersize{text margin left=1.2cm}
\setbeamersize{text margin right=1.2cm}

%%%%%%%%%%%%%%%%%%%%%%%%%%%%%%%%%%%%%%%%%%%%%%%%%%%%%%%%%%%%%%%%%%%%%%%%
%%% images
%%%%%%%%%%%%%%%%%%%%%%%%%%%%%%%%%%%%%%%%%%%%%%%%%%%%%%%%%%%%%%%%%%%%%%%%
\pgfdeclareimage[height=0.7\paperheight]{avrdip}{avr_dip}
\pgfdeclareimage[height=0.7\paperheight]{breadboard}{breadboard}
\pgfdeclareimage[height=0.3\paperheight]{skbottom}{starterkit_bottom}
\pgfdeclareimage[height=0.3\paperheight]{sktop}{starterkit_top}
\pgfdeclareimage[height=0.5\paperheight]{skparts}{starterkit_parts}
\pgfdeclareimage[height=0.3\paperheight]{skpcbs}{starterkit_pcbs}

%%%%%%%%%%%%%%%%%%%%%%%%%%%%%%%%%%%%%%%%%%%%%%%%%%%%%%%%%%%%%%%%%%%%%%%%
%%% title, author, date
%%%%%%%%%%%%%%%%%%%%%%%%%%%%%%%%%%%%%%%%%%%%%%%%%%%%%%%%%%%%%%%%%%%%%%%%
\title{AVR101}
\subtitle{OHM version}
\author{Netz39 e.\,V.}
\institute{\url{http://www.netz39.de/}}
\date{2013}
\subject{subj}
\keywords{AVR, Workshop}

%%%%%%%%%%%%%%%%%%%%%%%%%%%%%%%%%%%%%%%%%%%%%%%%%%%%%%%%%%%%%%%%%%%%%%%%
%%% document
%%%%%%%%%%%%%%%%%%%%%%%%%%%%%%%%%%%%%%%%%%%%%%%%%%%%%%%%%%%%%%%%%%%%%%%%
\begin{document}

\begin{frame}
	\titlepage
\end{frame}

\begin{frame}{Overview}
    \tableofcontents
\end{frame}

\section{Why?}

\subsection{Learning Things}

\begin{frame}{Learn All The Detailz}
    \begin{itemize}
        \item CPU
        \item Register
        \item Memory
        \item Interfaces and Ports
        \item \dots
    \end{itemize}
\end{frame}

\begin{frame}{Assembler and C}
    \begin{itemize}
        \item close to CPU
        \item efficient (if you're doing it right)
        \item available for microcontrollers
        \item standard in embedded world
        \item dependent on uC none or very few alternatives
        \item \dots
    \end{itemize}
\end{frame}

\subsection{Uses Cases}

\begin{frame}{Where?}
    \begin{itemize}
        \item Industry
        \item Home
        \item Multimedia
        \item Automotive
        \item Communications
        \item Space
        \item …
    \end{itemize}
\end{frame}

\begin{frame}{What?}
    \begin{itemize}
        \item Light Emitting Stuff
        \item Quadrocopter
        \item Human Interface Devices
        \item Home Control Foobar
        \item Measuring Devices
        \item Programmer
        \item \dots
    \end{itemize}
\end{frame}

\section{Needed Things}

\subsection{Microcontroller}

\begin{frame}{Families}
    \begin{itemize}
        \item Atmel AVR
        \item Microchip PIC
        \item Intel 8051
        \item C166/C167 (Siemens/Infineon)
        \item Renesas R8C/M16C/…
        \item TI MSP430
        \item …
    \end{itemize}
\end{frame}

\begin{frame}{ATtiny and ATmega}
    \begin{figure}
        \pgfuseimage{avrdip}
    \end{figure}
\end{frame}

\subsection{Circuits}

\begin{frame}{Breadboard}
    \begin{figure}
        \pgfuseimage{breadboard}
    \end{figure}
\end{frame}

\begin{frame}{Kits}
    \begin{figure}
        \pgfuseimage{skparts}
    \end{figure}
    \begin{itemize}
        \item \url{http://www.netz39.de/projekte/starterkit/}
        \pause
        \item AVR STK 500: $\approx$ 85\,€
    \end{itemize}
\end{frame}

%\begin{frame}{Atmel Developerkitz}
%    \begin{itemize}
%        \item{AVR DRAGON: $\approx$ 50\,€}
%        \item{AVR STK 500: $\approx$ 85\,€}
%    \end{itemize}
%\end{frame}

\begin{frame}{Arduino and other devices}
    \begin{itemize}
        \item works even for this workshop
        \item works with breadboard arduino clones
        \item works with your 2nd usbtiny programmer
        \item any circuit with AVR, LED and ISP header will do
    \end{itemize}
    \pause
    \begin{alertblock}{Warning}
        You'll destroy the bootloader/firmware of your device and have
        to restore it if you want to use it as before.
    \end{alertblock}
\end{frame}

\subsection{Programmer}

\begin{frame}{Programming Devices}
    \begin{itemize}
        \item{AVR ISP2: $\approx$ 40\,€}
        \item{AVR DRAGON: $\approx$ 50\,€}
        \item{JTAG ICE 3: $\approx$ 115\,€}
        \item{JTAG ICE MK II: $\approx$ 345\,€}
        \pause
        \item{serial (RS232)}
        \item{USB, e.\,g. \emph{usbtiny}}
        \item{any SPI-Device like Arduino, Raspberry Pi, r0ket, …}
    \end{itemize}
\end{frame}

%\begin{frame}{Other Programming Devices}
%    \begin{itemize}
%        \item …
%    \end{itemize}
%\end{frame}

\subsection{Software}

\begin{frame}{AVR Studio}
    This is M\$ Windoof, who uses this?

    \pause

    \emph{(Not covered in this workshop!)}
\end{frame}

\begin{frame}{Linux}
    \begin{block}{GUI anyone?}
        \emph{CLI rulez!}
    \end{block}
    \begin{itemize}
        \item gcc-avr
        \item binutils-avr
        \item avrdude
        \item avr-libc
        \item \$EDITOR
        \item any version control (Mercurial, Git, Subversion, …)
    \end{itemize}
\end{frame}

\section{Flashing the Device}

\subsection{avrdude}

\begin{frame}{avrdude}
    \begin{itemize}
        \item<1-5> \texttt{-v} -- verbose output
        \item<2-5> \texttt{-c programmer-id} -- e.\,g. \texttt{usbtiny} or \texttt{avrisp2}
        \item<3-5> \texttt{-p partno} -- e.\,g. \texttt{t85} or \texttt{m162}
        \item<4-5> \texttt{-P port} -- e.\,g. \texttt{usb}
        \item<5> \texttt{-t} -- test connection
        \item<6> see \texttt{`man avrdude`}
    \end{itemize}
\end{frame}

\subsection{All the Ugly Stuff}

\begin{frame}{Bootloader}
    \emph{This is advanced stuff!}
\end{frame}

\begin{frame}{Fuse Bits}
    \begin{itemize}
        \item \emph{not needed in this workshop}
        \pause
        \item \url{http://www.engbedded.com/fusecalc}
    \end{itemize}
\end{frame}

\begin{frame}[fragile]{Makefile}
    \begin{lstlisting}[
        language=make,
        basicstyle=\ttfamily\scriptsize\mdseries,
    ]
PGM = avrisp2
CPU_GCC = attiny25
CPU = t25

F_CPU = 1000000

CDEFS = -DF_CPU=$(F_CPU)
CFLAGS = -mmcu=$(CPU_GCC) $(CDEFS) -Wall -Os

.phony: clean

all: rgb.hex

program: rgb.hex
        avrdude -c $(PGM) -P usb -p $(CPU) -U flash:w:rgb.hex

clean:
        rm *.o *.elf *.hex

rgb.hex: rgb.c
        avr-gcc $(CFLAGS) -c rgb.c -o rgb.o
        avr-gcc $(CFLAGS) rgb.o -o rgb.elf
        avr-objcopy -R .eeprom -O ihex rgb.elf rgb.hex
    \end{lstlisting}
\end{frame}

\section{Hello World}

\subsection{LED Blinking}

\begin{frame}{Simple Approach}
\end{frame}

\begin{frame}{Pulse Width Modulation}
    \begin{itemize}
        \item on on on on off off on on on on
        \pause
        \item \emph{add nice plot here, TikZ anyone?}
        \pause
        \item not linear with perceived LED brightness
        \item don't fear pre calculated logarithmic lookup tables
    \end{itemize}
\end{frame}

\begin{frame}{Timers}
    \begin{itemize}
        \item count up or down
        \item count to a certain value
        \item count endlessly till overflow
        \item count exertnal events
        \pause
        \item you really need the micro controller documentation!
    \end{itemize}
\end{frame}

%\subsection{Input}

%\subsection{Red, Green, and Blue}

\section{Contact}

\begin{frame}{Contact}
    \begin{center}
        \begin{description}[Twitter/identi.ca]
            \item[WWW] \url{http://www.netz39.de/}
            \item[Twitter/identi.ca] @netz39
            \item[E-Mail] kontakt@netz39.de
            \item[Mailingliste] list@netz39.de
            \item[IRC] \#netz39 on freenode
        \end{description}
    \end{center}
\end{frame}

\appendix

\section{License}

\begin{frame}{License}
    % \begin{block}{Bilder}
        % Die Bilder sind unter folgender Creative Commons-Lizenz
        % veröffentlicht: \emph{Namensnennung-Keine kommerzielle
        % Nutzung-Weitergabe unter gleichen Bedingungen 3.0}, (CC-BY-NC-SA
        % 3.0).
    % \end{block}
    \begin{block}{Slides}
        This work is licensed under a \emph{Creative Commons
        Attribution-ShareAlike 3.0 Unported License}, (CC-BY-SA 3.0).
        Source at: \url{https://github.com/netz39/avr_101}
    \end{block}
\end{frame}



\end{document}
