%%%%%%%%%%%%%%%%%%%%%%%%%%%%%%%%%%%%%%%%%%%%%%%%%%%%%%%%%%%%%%%%%%%%%%%%
%%% documentclass and packages
%%%%%%%%%%%%%%%%%%%%%%%%%%%%%%%%%%%%%%%%%%%%%%%%%%%%%%%%%%%%%%%%%%%%%%%%
\RequirePackage{atbegshi}           % workaround for newer PGF versions
\documentclass{beamer}
% https://sourceforge.net/tracker/index.php?func=detail&aid=1848912&group_id=92412&atid=600660
\usepackage{lmodern}
\usepackage[T1]{fontenc}
\usepackage[utf8]{inputenc}
\usepackage{textcomp}
\usepackage[ngerman]{babel}
\usepackage[babel,english=american,german=guillemets]{csquotes}	% french
\usepackage{microtype}

% colors for listings
\definecolor{lightergray}{gray}{.95}
\definecolor{darkblue}{rgb}{0,0,0.5}
\definecolor{darkgreen}{rgb}{0,0.5,0}
\definecolor{darkred}{rgb}{0.5,0,0}
\definecolor{darkerblue}{rgb}{0,0,0.4}
\definecolor{darkergreen}{rgb}{0,0.4,0}
\definecolor{darkerred}{rgb}{0.4,0,0}

%\usepackage{listings}
%\lstloadlanguages{HTML,XML}
%\lstset{
%    basicstyle=\ttfamily\small\mdseries,
%    keywordstyle=\bfseries\color{darkblue},
%    identifierstyle=,
%    commentstyle=\color{darkgray},
%    stringstyle=\itshape\color{darkred},
%    frame=none,
%    showstringspaces=false,
%    tabsize=4,
%    backgroundcolor=\color{lightergray},
%}

%%%%%%%%%%%%%%%%%%%%%%%%%%%%%%%%%%%%%%%%%%%%%%%%%%%%%%%%%%%%%%%%%%%%%%%%
%%% preparations for beamer
%%%%%%%%%%%%%%%%%%%%%%%%%%%%%%%%%%%%%%%%%%%%%%%%%%%%%%%%%%%%%%%%%%%%%%%%
\useinnertheme{default}
\useoutertheme{infolines}
%\usecolortheme[rgb={0.28,0.37,0.52}]{structure}
\usecolortheme[rgb={0.18,0.23,0.33}]{structure}
%\usecolortheme{beaver}
\usefonttheme{structurebold}

%%% Ränder vergrößern für's Café Central
\setbeamersize{text margin left=1.2cm}
\setbeamersize{text margin right=1.2cm}

%%%%%%%%%%%%%%%%%%%%%%%%%%%%%%%%%%%%%%%%%%%%%%%%%%%%%%%%%%%%%%%%%%%%%%%%
%%% images
%%%%%%%%%%%%%%%%%%%%%%%%%%%%%%%%%%%%%%%%%%%%%%%%%%%%%%%%%%%%%%%%%%%%%%%%
%\pgfdeclareimage[height=0.75\paperheight]{strasse}{strasse}
%\pgfdeclareimage[height=0.75\paperheight]{arbeitsraum}{arbeitsraum}
%\pgfdeclareimage[height=0.75\paperheight]{lounge}{lounge}
%\pgfdeclareimage[height=0.75\paperheight]{mate}{mate}
%\pgfdeclareimage[height=0.75\paperheight]{regal}{regal}
%\pgfdeclareimage[height=0.75\paperheight]{werkstatt}{werkstatt}

%%%%%%%%%%%%%%%%%%%%%%%%%%%%%%%%%%%%%%%%%%%%%%%%%%%%%%%%%%%%%%%%%%%%%%%%
%%% title, author, date
%%%%%%%%%%%%%%%%%%%%%%%%%%%%%%%%%%%%%%%%%%%%%%%%%%%%%%%%%%%%%%%%%%%%%%%%
\title{AVR101}
\subtitle{OHM version}
\author{Netz39 e.\,V.}
\institute{\url{http://www.netz39.de/}}
\date{2013}
\subject{subj}
\keywords{AVR, Workshop}

%%%%%%%%%%%%%%%%%%%%%%%%%%%%%%%%%%%%%%%%%%%%%%%%%%%%%%%%%%%%%%%%%%%%%%%%
%%% document
%%%%%%%%%%%%%%%%%%%%%%%%%%%%%%%%%%%%%%%%%%%%%%%%%%%%%%%%%%%%%%%%%%%%%%%%
\begin{document}

\begin{frame}
	\titlepage
\end{frame}

\begin{frame}{Overview}
    \tableofcontents
\end{frame}

\section{Why?}

\subsection{Learning Things}

\begin{frame}{Learn All The Detailz}
    \begin{itemize}
        \item CPU
        \item Register
        \item Memory
        \item Interfaces and Ports
        \item \dots
    \end{itemize}
\end{frame}

\begin{frame}{Assembler and C}
    \begin{itemize}
        \item close to CPU
        \item efficient (if you're doing it right)
        \item available for microcontrollers
        \item standard in embedded world
        \item dependent on uC none or very few alternatives
        \item \dots
    \end{itemize}
\end{frame}

\subsection{Uses Cases}

\begin{frame}{Where?}
    \begin{itemize}
        \item Industry
        \item Home
        \item Multimedia
        \item Automotive
        \item Communications
        \item Space
        \item …
    \end{itemize}
\end{frame}

\begin{frame}{What?}
    \begin{itemize}
        \item Light Emitting Stuff
        \item Quadrocopter
        \item Human Interface Devices
        \item Home Control Foobar
        \item Measuring Devices
        \item Programmer
        \item \dots
    \end{itemize}
\end{frame}

\section{Needed Things}

\subsection{Microcontroller}

\begin{frame}{Families}
    \begin{itemize}
        \item Atmel AVR
        \item Microchip PIC
        \item Intel 8051
        \item C166/C167 (Siemens/Infineon)
        \item Renesas R8C/M16C/…
        \item TI MSP430
        \item …
    \end{itemize}
\end{frame}

\begin{frame}{ATtiny and ATmega}
    take pictures and add here \dots
\end{frame}

\subsection{Circuits}

\begin{frame}{Breadboard}
    Bild?
\end{frame}

\begin{frame}{Kits}
    Bild Netz39 Starterkit

    \url{http://www.netz39.de/projekte/starterkit/}

    \pause
    AVR STK 500: $\approx$ 85\,€
\end{frame}

%\begin{frame}{Atmel Developerkitz}
%    \begin{itemize}
%        \item{AVR DRAGON: $\approx$ 50\,€}
%        \item{AVR STK 500: $\approx$ 85\,€}
%    \end{itemize}
%\end{frame}

\begin{frame}{Arduino}
    \begin{itemize}
        \item Will work even for this workshop!
        \item \url{http://hackaday.com/2013/07/10/build-a-bare-bones-arduino-clone-which-maximizes-its-use-of-real-estate/}
    \end{itemize}
    \pause
    \begin{alertblock}{Warning}
        You'll destroy your bootloader and have to restore it if you
        want to use it as Arduino again.
    \end{alertblock}
\end{frame}

\subsection{Programmer}

\begin{frame}{Programming Devices}
    \begin{itemize}
        \item{AVR ISP2: $\approx$ 40\,€}
        \item{AVR DRAGON: $\approx$ 50\,€}
        \item{JTAG ICE 3: $\approx$ 115\,€}
        \item{JTAG ICE MK II: $\approx$ 345\,€}
        \pause
        \item{serial (RS232)}
        \item{USB, e.\,g. \emph{usbtiny}}
        \item{any SPI-Device like Arduino, Raspberry Pi, r0ket, …}
    \end{itemize}
\end{frame}

%\begin{frame}{Other Programming Devices}
%    \begin{itemize}
%        \item …
%    \end{itemize}
%\end{frame}

\subsection{Software}

\begin{frame}{AVR Studio}
    This is M\$ Windoof, who uses this?

    \pause

    \emph{(Not covered in this workshop!)}
\end{frame}

\begin{frame}{Linux}
    \begin{block}{GUI anyone?}
        \emph{CLI rulez!}
    \end{block}
    \begin{itemize}
        \item gcc-avr
        \item binutils-avr
        \item avrdude
        \item avr-libc
        \item \$EDITOR
        \item any version control (Mercurial, Git, Subversion, …)
    \end{itemize}
\end{frame}

\section{Flashing the Device}

\subsection{avrdude}

\subsection{All the Ugly Stuff}

\begin{frame}{Fuse Bits}
    \url{http://www.engbedded.com/fusecalc}
\end{frame}

\begin{frame}{Makefile}
\end{frame}

\begin{frame}{Bootloader}
    \emph{This is advanced stuff!}
\end{frame}

\section{Hello World}

\subsection{LED Blinking}

\begin{frame}{Simple Approach}
\end{frame}

\begin{frame}{PWM}
\end{frame}

\begin{frame}{Timers}
\end{frame}

\subsection{Input}

\subsection{Red, Green, and Blue}

\section{Contact}

\begin{frame}{Contact}
    \begin{center}
        \begin{description}[Twitter/identi.ca]
            \item[WWW] \url{http://www.netz39.de/}
            \item[Twitter/identi.ca] @netz39
            \item[E-Mail] kontakt@netz39.de
            \item[Mailingliste] list@netz39.de
            \item[IRC] \#netz39 on freenode
        \end{description}
    \end{center}
\end{frame}

\appendix

\section{License}

\begin{frame}{License}
    % \begin{block}{Bilder}
        % Die Bilder sind unter folgender Creative Commons-Lizenz
        % veröffentlicht: \emph{Namensnennung-Keine kommerzielle
        % Nutzung-Weitergabe unter gleichen Bedingungen 3.0}, (CC-BY-NC-SA
        % 3.0).
    % \end{block}
    \begin{block}{Slides}
        This work is licensed under a \emph{Creative Commons
        Attribution-ShareAlike 3.0 Unported License}, (CC-BY-SA 3.0).
        Source at: \url{https://github.com/netz39/avr_101}
    \end{block}
\end{frame}



\end{document}
